\documentclass[]{article}
\usepackage{geometry}[margin=1in]
\usepackage{amsmath}
\usepackage{physics}
\usepackage{graphicx}
\usepackage{cancel}
\usepackage{setspace}

\setlength\parindent{0pt}

\newcommand{\sectionname}{Section}


%opening
\title{MECH 6300 - Problem Set A}
\author{Jonas Wagner}
\date{2020, August 21}



\begin{document}

\maketitle


\section{Problem 1}

	Using the Lagrangian function of Design Application 2, compute the Lagrange equations for the system.
	
	\subsection{Design Application 2 Background}
	
		Design Application 2 is that of an inverted pendulum. This consists of a mass on a rod that extends upward from a movable cart that can be balanced at its upright equilibrium point.
		
		\begin{figure}[h]
			\centering
			\includegraphics[width=0.5\linewidth]{Fig/DesignApplication2}
			\caption[]{Inverted Pendulum Mechanics Figure}
			\label{fig:designapplication2}
		\end{figure}
		
		Given the diagram, \figurename \ \ref{fig:designapplication2}, the following parameters are defined:
		\begin{align*}
			M &\equiv \text{Mass of the cart}\\
			m &\equiv \text{Mass at end of the rod}\\
			l &\equiv \text{Length of the rod}
		\end{align*}
		
		The two primary coordinates of the system are:
		\begin{align*}
			z &\equiv \text{Cart z-position}\\
			\theta &\equiv \text{Pendulum angle}
		\end{align*}
		
		Additional variables are defined in the figure:
		\begin{align*}
			h &\equiv \text{Mass height}\\
			\bar{z} &\equiv \text{Mass z-position}
		\end{align*} 
		
		These variables can be related to the primary coordinates as such:
		\begin{align}
			h		&= l \cos(\theta) \label{eq:h_def}\\
			\bar{z} &= z + l \sin(\theta) \label{eq:z_def}	
		\end{align}
		
		The derivatives of each of these variables can also be computed as such:
		\begin{align}
			\dot{h} &= -l \dot{\theta} \sin(\theta) \label{eq:h_dot}\\
			\dot{\bar{z}} &= \dot{z} + l \dot{\theta}\cos(\theta) \label{eq:z}
		\end{align}
		
		The total kinetic energy, $E_k$, can be defined as the sum of the kinetic energy of the cart and pendulum mass:
		\begin{align}
			E_k &= \frac{1}{2} M \dot{z}^2 + \frac{1}{2} m \qty(\sqrt{\dot{\bar{z}}^2 + \dot{h}^2})^2 \label{eq:E_k_original}\\
				&= \frac{1}{2} M \dot{z}^2 + \frac{1}{2} m \qty(\qty(\dot{z} + l \dot{\theta}\cos(\theta))^2 + \qty(-l \dot{\theta} \sin(\theta))^2)\nonumber\\
				&= \frac{1}{2} M \dot{z}^2 + \frac{1}{2} m \qty(\qty(\dot{z}^2 + 2 l \dot{z} \dot{\theta}\cos(\theta) + l^2 \dot{\theta}^2 \cos[2](\theta)) + \qty(l^2 \dot{\theta}^2 \sin[2](\theta)))\nonumber\\
				&= \frac{1}{2} M \dot{z}^2 + \frac{1}{2} m \qty(\dot{z}^2 + 2 l \dot{z} \dot{\theta}\cos(\theta) + l^2 \dot{\theta}^2 \qty(\cos[2](\theta)\sin[2](\theta)))\nonumber
		\end{align}
		
		This results in a simplified kinetic energy equation of:
		\begin{equation}
				E_k = \frac{1}{2} (M + m) \dot{z}^2 + m l \dot{z} \dot{\theta}\cos(\theta) + \frac{1}{2} l^2 \dot{\theta}^2
				\label{eq:E_k_simplified}
		\end{equation}
		
		The potential energy, $E_p$, consists only of the gravitational potential energy within the pendulum mass:
		\begin{align}
			E_p &= m g h\\
			E_p &= m g l \cos(\theta) \label{eq:E_p}
		\end{align}
		
		The Lagrangian can then be defined by the difference between the total kinetic \eqref{eq:E_k_simplified} and potential \eqref{eq:E_p} energy as follows:
		\begin{align}
			L &= E_k - E_p\\
			L &= \frac{1}{2} (M + m) \dot{z}^2 + m l \dot{z} \dot{\theta}\cos(\theta) + \frac{1}{2} ml^2 \dot{\theta}^2 - m g l \cos(\theta) \label{eq:L}
		\end{align}
	
	\subsection{$z$-coordinate Lagrange Equation}
	
		Utilizing the Lagrangian of the inverted pendulum system \eqref{eq:L}, the Z-coordinate Lagrange equation can be found using the following equation:
		\begin{equation}
			\dv{t} \qty(\pdv{L}{\dot{z}}) - \pdv{L}{z} = f
			\label{eq:lagrange_Z_original}
		\end{equation}
		
		First, the partial derivatives can be calculated as follows:
		\begin{align}
			\pdv{L}{\dot{z}} &= (M+m)\dot{z} + m l \dot{\theta}\cos(\theta) \label{eq:pdv_L_z_dot}\\
			\pdv{L}{z} &= 0 \label{eq:pdv_L_z}
		\end{align}
		
		The time derivative can then be computed from \eqref{eq:pdv_L_z_dot}:
		\begin{equation}
			\dv{t} \qty(\pdv{L}{\dot{z}}) = (M+m)\ddot{z} + m l \qty(\ddot{\theta}\cos(\theta) - \dot{\theta}^2\sin(\theta))
			\label{eq:dv_pdv_L_z_dot}
		\end{equation}
		
		The Lagrange equation can then be derived using \eqref{eq:lagrange_Z_original}, \eqref{eq:pdv_L_z}, and \eqref{eq:dv_pdv_L_z_dot}:
		\begin{equation}
			(M+m)\ddot{z} + m l \qty(\ddot{\theta}\cos(\theta) - \dot{\theta}^2\sin(\theta)) = f
			\label{eq:lagrange_Z}
		\end{equation}
	
	\subsection{$\theta$-coordinate Lagrange Equation}
	
		Utilizing the Lagrangian of the inverted pendulum system \eqref{eq:L}, the $\theta$-coordinate Lagrange equation can be found using the following equation:
		\begin{equation}
			\dv{t} \qty(\pdv{L}{\dot{\theta}}) - \pdv{L}{\theta} = 0
			\label{eq:lagrange_theta_original}
		\end{equation}
		
		First, the partial derivatives can be calculated as follows:
		\begin{align}
			\pdv{L}{\dot{\theta}} &= m l \dot{z} \cos(\theta) + m l^2 \dot{\theta} \label{eq:pdv_L_theta_dot}\\
			\pdv{L}{\theta} &= - m l \dot{z} \dot{\theta} \sin(\theta) + m g l \sin(\theta) \label{eq:pdv_L_theta}
		\end{align}
		
		The time derivative can then be computed from \eqref{eq:pdv_L_theta_dot}:
		\begin{equation}
			\dv{t} \qty(\pdv{L}{\dot{\theta}}) = m l^2 \ddot{\theta} + m l \ddot{z} \cos(\theta) - m l \dot{z} \dot{\theta} \sin(\theta)
			\label{eq:dv_pdv_L_theta_dot}
		\end{equation}
		
		The Lagrange equation can then be derived using \eqref{eq:lagrange_theta_original}, \eqref{eq:pdv_L_theta}, and \eqref{eq:dv_pdv_L_theta_dot}:
		\begin{align}
			 \qty(m l^2 \ddot{\theta} + m l \ddot{z} \cos(\theta) - m l \dot{z} \dot{\theta} \sin(\theta)) - \qty(- m l \dot{z} \dot{\theta} \sin(\theta) + m g l \sin(\theta))= 0 \nonumber\\
			 m l^2 \ddot{\theta} - m g l \sin(\theta) + m l \ddot{z} \cos(\theta) = 0
			\label{eq:lagrange_theta}
		\end{align}

	\subsection{Solution:}
	
		The two Lagrange equations computed were computed in \eqref{eq:lagrange_Z} and \eqref{eq:lagrange_theta}:
		\begin{align}
			(M+m)\ddot{z} + m l \qty(\ddot{\theta}\cos(\theta) - \dot{\theta}^2\sin(\theta)) &= f  \tag{\ref{eq:lagrange_Z}}\\
			m l^2 \ddot{\theta} - m g l \sin(\theta) + m l \ddot{z} \cos(\theta) &= 0
			\tag{\ref{eq:lagrange_theta}}
		\end{align}

\section{Problem 2}
	Derive a linearized model at the upright equilibrium point using the following approximations:
	\begin{align*}
		\cos(\theta) 	&\approx 1\\
		\sin(\theta) 	&\approx \theta\\
		\dot{z}^2		&\approx 0\\
		\dot{\theta}^2  &\approx 0
	\end{align*}
	
	\subsection{$z$-coordinate Liniarization}
		From \eqref{eq:lagrange_Z}, the following can be derived:
		\begin{align}
			(M+m)\ddot{z} &= f -ml \qty(\ddot{\theta}\cancelto{1}{\cos{\theta}} - \cancelto{0}{\dot{z}^2}\cancelto{\theta}{\sin{\theta}} ) \nonumber\\
			\ddot{z} &= \frac{f -ml \ddot{\theta}}{M+m} \label{eq:z_lin_init}
		\end{align}
	
	\subsection{$\theta$-coordinate Linearization}
		From \eqref{eq:lagrange_theta}, the following can be derived:
		\begin{align}
			m l^2 \ddot{\theta} &= m g l \cancelto{\theta}{\sin{\theta}} - m l \ddot{z} \cancelto{1}{\cos{\theta}} \nonumber\\
			\ddot{\theta} &= \frac{g\theta -\ddot{z}}{l} \label{eq:theta_lin_init}
		\end{align}
		
	\subsection{Substitution and Simplification}
		By substituting \eqref{eq:theta_lin_init} into \eqref{eq:z_lin_init}, the following can be obtained:
		\begin{align}
			\ddot{z} &= \frac{f -m\bcancel{l} \qty(\cfrac{g\theta-\ddot{z}}{\bcancel{l}})}{M+m} \nonumber\\
			\ddot{z} \qty(1-\frac{m}{M+m}) &= \frac{f - mg\theta}{M+m} \nonumber\\
			\ddot{z} &= \frac{f-mg\theta}{\qty(1-\cfrac{m}{M+m}) \qty(M+m)} \nonumber\\
			\ddot{z} &= \frac{f-mg\theta}{M} \label{eq:z_lin}
		\end{align}
		
		\eqref{eq:theta_lin_init} can then rewritten as:
		\begin{align}
			\ddot{\theta} &= \frac{g\theta -\qty(\cfrac{f-mg\theta}{M})}{l} \nonumber\\
			\ddot{\theta} &= \frac{(m+M)g\theta - f}{Ml} \nonumber\\
			\ddot{\theta} &= \frac{(m+M)g\theta}{Ml} - \frac{f}{Ml} \label{eq:theta_lin}
		\end{align}

\newpage	
\section{Problem 3}
	Put the system into standard variable form:
	\begin{equation}
		\dot{\vb*{x}} = A \vb*{x} + B \vb*{u} \label{eq:state_eq_def}
	\end{equation}
	
	where $\vb*{x}$ is the following state vector:
	\begin{equation}
		\vb*{x} = \mqty[z \\ \theta \\ \dot{z} \\ \dot{\theta}] \label{eq:state_var}
	\end{equation}
	
	and the input $\vb*{u}$ is defined as:
	\begin{equation}
		\vb*{u} = f
	\end{equation}
	
	\subsection{State-variable Equations}
	The state-equations and  linearized equations \eqref{eq:z_lin} and \eqref{eq:theta_lin} can be used to generate the following state-variable equations:
	\begin{equation}
		\begin{aligned}
			\dot{z} &= \dot{z} = \vb*{x}[3]\\
			\dot{\theta} & = \dot{\theta} = \vb*{x}[4]\\
			\ddot{z} & = \frac{-mg}{M}\vb*{x}[3] + \frac{1}{M} \vb*{u}[1]\\
			\ddot{\theta} & = \frac{(m+M)g}{Ml}\vb*{x}[3] - \frac{1}{Ml} \vb*{u}[1]
		\end{aligned}
		\label{eq:state_eq_init}
	\end{equation}
	
	\subsection{Defining the State-matrices}
	From the state-equations \eqref{eq:state_eq_init}, the following state-matrices can be derived:
	\setstretch{1.2}
	\begin{equation}
		\begin{aligned}
			&A = \mqty[	0 & 0 & 1 					& 0\\
						0 & 0 & 0 					& 1\\
						0 & 0 & \cfrac{-mg}{M} 		& 0\\
						0 & 0 & \cfrac{(m+M)g}{Ml}	& 0]
			&B = \mqty[	0\\
						0\\
						\cfrac{1}{M}\\
						\cfrac{-1}{Ml}]
		\end{aligned}
		\label{eq:state_matrices}
	\end{equation}
	\setstretch{1}

\newpage
\section{Problem 4} \label{sec:p4}
	Using the background provided for design application 1, derive the overall system of the inverted pendulum on a cart with a DC motor driving wheels to move the cart.
	
	\subsection{Design Application 1 Background}
		Design Application 1 describes a DC motor with a load. The system consists of a DC motor with an inertial load, $J$, that converts a voltage input, $e$, into a radial position, $\theta$.
		
		\begin{figure}[h]
			\centering
			\includegraphics[width=0.7\linewidth]{Fig/DesignApplication1}
			\caption{DC Motor with Load Diagram}
			\label{fig:designapplication1}
		\end{figure}
		
		The following parameters are defined to describe the DC motor and load operation:
		\begin{align*}
			J &\equiv \text{Inertial Load}\\
			R &\equiv \text{Armature Resistance}\\
			K_1 &\equiv \text{Torque-Current Motor Constant}\\
			K_2 &\equiv \text{Voltage-Speed Motor Constant}
		\end{align*}
		
		The two primary State variables are defined as:
		\begin{align*}
			e &\equiv \text{Voltage Input}\\
			\theta &\equiv \text{Radial Position}
		\end{align*}
		
		Additional variables are defined as:
		\begin{align*}
			\tau 	&\equiv \text{Torque}\\
			i		&\equiv \text{Input Current}\\
			v 		&\equiv \text{Back EMF Voltage}\\
			\omega	&\equiv \text{Output Rotational Velocity}
		\end{align*}
		
		Given the physics of a DC brushed motor, the following relationships exist:
		\begin{align}
			\tau &= K_1 i \label{eq:tau_i_rel}\\
			v &= K_2 \omega \label{eq:v_omega_rel}\\
		\end{align}
		
		Additionally, with the assumption of 100\% \ efficiency, the following can be stated:
		\begin{align}
			k = K_1 &= K_2 \label{eq:K1_K2}
		\end{align}
		
		From Ohm's Law, the following is known:
		\begin{align}
			e - v &= Ri \nonumber\\
			i &= \frac{e-v}{R} \label{eq:armature_Ohms}
		\end{align}
		
		From rotational dynamics it is also known that:
		\begin{align}
			\tau &= J \dot{\omega} \label{eq:tau_omega_rel}\\
			\dot{\theta} &= \omega \label{eq:omega_def}
		\end{align}
		
		By equating \eqref{eq:tau_i_rel} and \eqref{eq:tau_omega_rel}, and then substituting \eqref{eq:armature_Ohms} and \eqref{eq:v_omega_rel}, the following can be derived:
		\begin{align}
			 J \dot{\omega}	&= \tau = K_1 i \nonumber\\
			 J \dot{\omega}	&= K_1 \qty(\frac{e-v}{R}) \nonumber\\
			 J \dot{\omega}	&= \frac{K_1 \qty(e - \qty(K_2 \omega))}{R} \nonumber\\
			 \dot{\omega}	&= \frac{K_1}{JR} e - \frac{K_1 K_2}{JR} \omega \label{eq:omega_state_eq}
		\end{align}
		
		The two state equations, \eqref{eq:omega_def} and \eqref{eq:omega_state_eq}, can then be rewritten in the standard state variable format:
		\begin{align}
			\mqty[	\dot{\theta}\\
					\dot{\omega}]
					&= \mqty[	0	&1\\
								0	&-\frac{K_1 K_2}{J R}]
						\mqty[	\theta\\
								\omega]
						+ \mqty[0\\
								\frac{K_1}{J R}] e \label{eq:DesignApp2_state_eq}
		\end{align}
	
	\newpage	
	\subsection{Implementation Into Cart}\label{sec:p4_2}
		The electric motor can be introduced into the inverted pendulum cart by relating the output torque to the force excreted at the wheels.
		
		The relationships between rotational and linear movement are defined as:
		\begin{align}
			z &= r \theta \label{eq:z_theta_rel}\\
			\dot{z} &= r \omega \label{eq:zdot_omega_rel}\\
			f &= \frac{\tau}{r} \label{eq:f_tau_rel}
		\end{align}
		where $r$ is the radius of the wheel.
		
		Using the primary state equation for the DC motor, \eqref{eq:omega_state_eq}, torque output from the motor can be derived as:
		\begin{align}
			\tau &= J \dot{\omega} = \frac{K_1}{R} e - \frac{K_1 K_2}{R} \omega \label{eq:torque_eq}
		\end{align}
		
		Using the assumption \eqref{eq:K1_K2} and the relationships \eqref{eq:z_theta_rel}, \eqref{eq:zdot_omega_rel}, and \eqref{eq:f_tau_rel}, \eqref{eq:torque_eq} can be converted to the linear equivalent:
		\begin{align}
			f r &= \frac{k}{R} e - \frac{k^2}{R} \frac{\dot{z}}{r} \nonumber\\
			f &= \frac{k}{R r} e - \frac{k^2}{r^2} \dot{z} \label{eq:f_def}
		\end{align}
		
		This can then be substituted into \eqref{eq:z_lin} and \eqref{eq:theta_lin} to create the overall system equations of motion:
		\begin{align}
			\ddot{z} &= - \frac{mg}{M} \theta + \frac{1}{M} \qty(\frac{k}{Rr}e - \frac{k^2}{R r^2} \dot{z}) \label{eq:z_pend_eq}\\
			\ddot{\theta} &= \frac{(M+m)g}{M l} \theta - \frac{1}{M l} \qty(\frac{k}{R r} e - \frac{k^2}{R r^2}\dot{z}) \label{eq:theta_pend_eq}
		\end{align}
	
\newpage
\section{Problem 5}
	Using the background provided in the design application 3, derive the state variable equations for the entire dual-cart system.
	\subsection{Design Application 3 Background}
		Design Application 3 describes a system of two coupled carts. The system consists of two carts of mass $M_1$ and $M_2$ that are connected by a spring with constant $K$.
		
		\begin{figure}[h]
			\centering
			\includegraphics[width=0.7\linewidth]{Fig/DesignApplication3}
			\caption{Coupled Carts Figure}
			\label{fig:designapplication3}
		\end{figure}
		
		The following parameters are defined to describe the DC motor and load operation:
		\begin{align*}
			M_1 &\equiv \text{Mass of Cart 1}\\
			M_2 &\equiv \text{Mass of Cart 2}\\
			K &\equiv \text{Spring Constant}
		\end{align*}
		
		The two primary state variables are defined as:
		\begin{align*}
			z_1 &\equiv \text{Cart 1 Position}\\
			z_2 &\equiv \text{Cart 2 Position}
		\end{align*}
		
		The two inputs are defined as:
		\begin{align*}
			F_1 &\equiv \text{Cart 1 Force}\\
			F_2 &\equiv \text{Cart 2 Force}
		\end{align*}
		
		The Lagrangean of the system containing no potential energy is defined solely by the total kinetic energy:
		\begin{align}
			L &= \frac{1}{2} \qty(M_1 \dot{z_1}^2 + M_2 \dot{z_2}^2) \label{eq:carts_lagrange}
		\end{align}
		
		Additionally, the force on excreted by the spring on each cart is equal and opposite:
		\begin{align}
			F_s &= K(z_2 - z_1) = F_{s1} = F_{s2} \label{eq:spring_force}
		\end{align}
		
		The lagrangian equations for $z_1$ and $z_2$ can then be defined as:
		\begin{align}
			\dv{t}\qty(\pdv{L}{\dot{z_1}}) - \pdv{L}{z_1} &= F_1 + F_s \label{eq:carts_lagrange_eq1}\\
			\dv{t}\qty(\pdv{L}{\dot{z_2}}) - \pdv{L}{z_2} &= F_2 - F_s \label{eq:carts_lagrange_eq2}
		\end{align}	
		
		The dynamics can then be easily calculated from \eqref{eq:carts_lagrange_eq1} and \eqref{eq:carts_lagrange_eq2} as:
		\begin{align}
			M_1 \ddot{z_1} &= F_1 + K(z_2 - z_1) \label{eq:carts_state_eq1}\\
			M_2 \ddot{z_2} &= F_2 - K(z_2 - z_1) \label{eq:carts_state_eq2}
		\end{align}
		
	\subsection{Integration of Motors}
		Integration of motors into the coupled carts system can be done by relating the output torque of the motors to the force excreted by the wheels of each cart.\\
		
		First, the following parameters for the motors can be defined as:
		\begin{align*}
			k_i &\equiv \text{Motor Torque Constant}\\
			R_i &\equiv \text{Motor Armature Resistance}\\
			r_i &\equiv \text{Motor Tourque-Force Relationship}\\
		\end{align*}
		
		The following variables are also defined for each motor:
		\begin{align*}
			\tau_i &\equiv \text{Motor Output Torque}\\
			f_i &\equiv \text{Motor Output Force}\\
			e_i &\equiv \text{Motor Applied Voltage}
		\end{align*}
		
		From calculations done in \sectionname \ \ref{sec:p4_2}, specifically \eqref{eq:f_def}, it is known that for each motor:
		\begin{align}
			F_i &= \frac{k_i}{R_i r_i} e_i - \frac{k_i^2}{R_i r_i^2} \dot{z_i} \label{eq:carts_f_def}
		\end{align}

		
	\subsection{Cart Dynamics}
		The dynamics of the first cart can be described by substituting \eqref{eq:carts_f_def} into \eqref{eq:carts_state_eq1}:
		\begin{align}
			M_1 \ddot{z_1} &= \frac{k_1}{R_1 r_1} e_1 - \frac{k_1^2}{R_1 r_1^2} \dot{z_1} + K(z_2 - z_1) \nonumber\\
			\ddot{z_1} &= -\frac{K}{M_1} z_1 + \frac{K}{M_1} z_2 - \frac{k_1^2}{M_1 R_1 r_1^2} \dot{z_1} + \frac{k_1}{M_1 R_1 r_1} e_1
		\end{align}
	
		The dynamics of the second cart can be described by substituting \eqref{eq:carts_f_def} into \eqref{eq:carts_state_eq2}:
		\begin{align}
			M_2 \ddot{z_2} &= \frac{k_2}{R_2 r_2} e_2 - \frac{k_2^2}{R_2 r_2^2} \dot{z_2} - K(z_2 - z_1) \nonumber\\
			\ddot{z_2} &= \frac{K}{M_2} z_1 - \frac{K}{M_2} z_2 - \frac{k_2^2}{M_2 R_2 r_2^2} \dot{z_2} + \frac{k_2}{M_2 R_2 r_2} e_2
		\end{align}
	
		With the state vector, $x$, and input vector, $u$, defined as:
		\begin{align}
			x &= \mqty[z_1\\ z_2\\ \dot{z_1}\\ \dot{z_2}] \label{eq:carts_x_def}\\
			\nonumber\\
			u &= \mqty[e_1\\ e_2] \label{eq:carts_u_def}
		\end{align}
		
		the state equations can be defined as:
		\begin{equation}\label{eq:carts_state_eqs}
			\begin{aligned}
				\dot{x}[1] &= x[3]\\
				\dot{x}[2] &= x[4]\\
				\dot{x}[3] &= -\frac{K}{M_1} x[1] + \frac{K}{M_1} x[2] - \frac{k_1^2}{M_1 R_1 r_1^2} x[3] + \frac{k_1}{M_1 R_1 r_1} u[1]\\
				\dot{x}[4] &= \frac{K}{M_2} x[1] - \frac{K}{M_2} x[2] - \frac{k_2^2}{M_2 R_2 r_2^2} x[4] + \frac{k_2}{M_2 R_2 r_2} u[2]
			\end{aligned}
		\end{equation}

	\subsection{Output Equation}
		The output equation is a function of the state vector and is not directly dependent on the input. With the output vector, $y$, defined as:
		\setstretch{1.25}
		\begin{align}
			y &= \mqty[z_1\\ z_2\\ \cfrac{\dot{z_1}}{r_1}\\ \cfrac{\dot{z_2}}{r_2}]\label{eq:carts_y_def}
		\end{align}
		\setstretch{1}
		
		the output equations can be defined as:
		\begin{equation}\label{eq:carts_y_ouput_eqs}
			\begin{aligned}
				y[1] &= x[1]\\
				y[2] &= x[2]\\
				y[3] &= \frac{1}{r_1} x[3]\\
				y[4] &= \frac{1}{r_2} x[4]
			\end{aligned}
		\end{equation}
	
	\subsection{State-Space Matrix Formulation}
		The coupled carts system can be put in the standard state-space formulation:
		\begin{equation}\label{eq:state_space_eqs_def}
			\begin{aligned}
				\dot{x} &= A x + B u\\
				y &= C x + D u
			\end{aligned}
		\end{equation}
		 
		 by defining the state matrices based on \eqref{eq:carts_state_eqs} and \eqref{eq:carts_y_ouput_eqs}.
		\begin{align}
			A &= \mqty[	0	& 0			& 1					& 0\\
						0	& 0	 		& 0					& 1\\
						-\cfrac{K}{M_1}	& \cfrac{K}{M_1}	& -\cfrac{k_1^2}{M_1 R_1 r_1^2}	& 0\\
						\cfrac{K}{M_2}	& -\cfrac{K}{M_2}	& 0								& -\cfrac{k_2^2}{M_2 R_2 r_2^2}
						]\label{eq:carts_A_matrix}\\
			\nonumber\\
			B &= \mqty[ 0						& 0\\
						0						& 0\\
						\cfrac{k_1}{M_1 R_1 r_1}	& 0\\
						0						& \cfrac{k_2}{M_2 R_2 r_2}
						]\label{eq:carts_B_matrix}\\
			\nonumber\\
			C &= \mqty[\dmat[0]{1,1,\cfrac{1}{r_1},\cfrac{1}{r_2}}]\label{eq:carts_C_matrix}\\
			\nonumber\\
			D &= \mqty[\zmat{4}{2}]\label{eq:carts_D_matrix}
		\end{align}
		
				
		
\end{document}
